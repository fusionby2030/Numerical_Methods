\documentclass{article}
\usepackage[utf8]{inputenc}
\usepackage{graphicx}
\usepackage{amsmath}
\usepackage{hyperref}
\usepackage[a4paper, textwidth=450pt]{geometry}

\graphicspath{ {./data/} }

\title{TP4 Homework 4}
\author{Adam Kit}
\date{4 May 2020}


\begin{document}

\maketitle
\section*{Fourier Transform and Gaussian Wave Functions}
Fourier transform and Gaussian wave functions.
\section{Fourier Transform}

Determine the Fourier transform of $$ \Psi (x) - (\pi \omega^2)^{-1/4}e^{ip_0x/\hbar - (x-x_0)^2/(2\omega_0^2)}$$ where $x_0$, $p_0$, $w_0$ are real parameters.
\subsection*{Solution}

\section{Normaliziation}
(b)  Show that $\psi$ is normalized $ $
\section{Mean Values}

\section{Adjoints}

\section{Variance}
For the wave functionΨin (a), show that the variances of position and momentum oper-ator are∆ˆx=|w0|√2,∆ˆp= ̄h|w0|√2.(5)(
\section{Differentiabl Wave Function}
f)  For any differentiable wave functionΦhaving sufficiently rapid decay as|x|→∞(aswell as its derivative), show that ̃(xΦ)(k) =−iddk ̃Φ(k), ̃(−iddxΦ)(k) =k ̃Φ(k),(6)so Fourier transform exchanges position and momentum operators. Show that ifΦis anL2-wave function, thenΨ(x) =eik0xΦ(x−x0)has Fourier transform ̃Ψ(k) =e−ix0k ̃Φ(k−k0). How are these statements related?
\section{Hamiltonian for Schrodinger Equation}
We now consider the Hamilton operatorˆH=12mˆp2=− ̄h22md2dx2for a free particle on thereal line. We wish to solve the time-dependent Schrodinger equation ̄hiddtΨt(x) =ˆHΨt(x),Ψ0(x) =Ψ(x)(7)with initial wave functionΨ(x)as in (a).  Using (e), show that such a solution satisfiesddt ̃Ψt(k) =−ik2 ̄h2m ̃Ψt(k). Using this and the result of (a), find ̃Ψt(k). Applying the inverseFourier transform, findΨt(x).
\section{General Solution Schrodinger Equation}

Using  the  same  type  of  method  as  in  (g),  show  that  the  general  solution  to  the  time-dependent Schrodinger equation for our Hamiltonian may be written asΦt(x) =∫dy Kt(x−y)Φ0(y),Kt(x) =(mi ̄h2t)1/2eimx22 ̄ht(8)Hint:  You may use the convolution theorem:  Iff(x)andΦ(x)are decaying sufficientlyrapidly, then ̃(fΦ)(k) =∫dp f(k−p)Φ(p),(9)together with a similar theorem for the inverse Fo


\end{document}
